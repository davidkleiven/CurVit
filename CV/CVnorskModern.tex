\documentclass[verdana]{moderncv}

\usepackage[scale=0.8]{geometry}
\moderncvstyle{classic}
\moderncvcolor{blue}
\usepackage{filecontents}
\usepackage{fontspec}
\usepackage{enumitem}
\setlist[itemize]{noitemsep, topsep=2pt}

\defaultfontfeatures{Ligatures=TeX}
\definecolor{linkColor}{rgb}{0.156 0.546 0.742}

\newcommand\Colorhref[3][cyan]{\href{#2}{\small\color{#1}#3}}
\newcommand{\projsep}{\par\smallskip\noindent}
\makeatletter
\renewcommand*{\bibliographyitemlabel}{\@biblabel{\arabic{enumiv}}}
\makeatother

\begin{filecontents}{conference.bib}
@article{kleiven2020precipitate,
  title={Precipitate Formation in Aluminum Alloys: Multi-scale Modeling Approach},
  author={Kleiven, David and Akola, Jaakko},
  journal={Acta Materialia},
  year={2020},
  publisher={Elsevier}
  }

  @article{chang2019clease,
  title={CLEASE: a versatile and user-friendly implementation of cluster expansion method},
  author={Chang, Jin Hyun and Kleiven, David and Melander, Marko and Akola, Jaakko and Garcia-Lastra, Juan Maria and Vegge, Tejs},
  journal={Journal of Physics: Condensed Matter},
  volume={31},
  number={32},
  pages={325901},
  year={2019},
  publisher={IOP Publishing}
}

  @article{kleiven2019atomistic,
  title={Atomistic simulations of early stage clusters in AlMg alloys},
  author={Kleiven, David and {\O}deg{\aa}rd, Olve L and Laasonen, Kari and Akola, Jaakko},
  journal={Acta Materialia},
  volume={166},
  pages={484--492},
  year={2019},
  publisher={Elsevier}
}


  @inproceedings{kleiven2016experimental,
  title={Experimental Setup to Measure the Damage Limits of Superconducting Magnets due to Beam Impact at CERN's HiRadMat Facility},
  author={Kleiven, David and Auchmann, Bernhard and Raginel, Vivien and Schmidt, Ruediger and Verweij, Arjan and Wollmann, Daniel},
  booktitle={7th International Particle Accelerator Conference (IPAC'16), Busan, Korea, May 8-13, 2016},
  pages={4200--4202},
  year={2016},
  organization={JACOW, Geneva, Switzerland}
  }

  @inproceedings{raginel2016degradation,
    title={Degradation of the insulation of the LHC main dipole cable when exposed to high temperatures},
    author={Raginel, Vivien and Auchmann, Bernhard and Kleiven, David and Schmidt, Ruediger and Verweij, Arjan and Wollmann, Daniel},
    booktitle={7th International Particle Accelerator Conference (IPAC'16), Busan, Korea, May 8-13, 2016},
    pages={1186--1189},
    year={2016},
    organization={JACOW, Geneva, Switzerland}
  }

\end{filecontents}

\firstname{David}
\lastname{Kleiven}
\title{Curriculum Vitae}
\address{Brøttemsåsvegen 5}{7540 Klæbu}
\mobile{93060916}
\email{davidkleiven446@gmail.com}
\photo[64pt]{../imageCrop.jpg}

\begin{document}
\makecvtitle

\section{Arbeidserfaring}
\cventry{2021 - 2026}{Data Engineer i Statnett}{}{}{}{
I Statnatt jobbet på ulike prosjekter innen programvare utvikling.\projsep
\textbf{Monster} --  Monte Carlo basert simuleringsverktøy for å kunne gjøre risikobasert vurderinger av utfall og tilhørende kostnader ved tilknytning til nytt forbruk og produksjon.
Oppgaver bestod stortsett av å utvikle analysemodulen i \emph{Monster} for datavisualisering, samt og vedlikeholde baysianske justeringer av sannsynilghetsfordelinger basert på innkommende feilhendelser. \projsep
\textbf{ABOT} -- Avoiding Bottlenecks prosjektet er et stort prosjekt som var viktig i forbindelse med overgang til 15 minutters oppløsning i balansemarkedet.
Systemet håndterer flaskehalser i strømnettet automatisk gjennom å utilgjengeliggjøre bud som med all sannsynlighet vil føre til overlaster, samt å
gjøre proaktiv systemregulering for å forhindre overlaster.\projsep
I Abot jobbet jeg med å utvikle optimeringsalgoritmen for optimal lastflyt samt å bygge nettoplogier fra data beskrevet innen Common Information Model (CIM) rammeverket.
Systemet håndterer også live topologioppdatering basert på utfall, operatørstyrte koblingsoppdrag og systemvernutløsning, og sikrer at arbeidshverdagen for operatør blir enklere.\projsep
\textbf{GridLens} -- Statnett har kjørte et utredningsarbeid med mål om å utredemuligheten for å utnytte informasjonen i CIM modeller mer direkte som databerikningskilde for å strømmende tidsserier.
	I dette arbeidet utviklet vi API-er som gir enkelt gir metadatainformasjon til tidsserier direkte fra en grafdatabase.
}


\cventry{2015 - 2016}{Teknisk student ved CERN}{}{}{}{
Jobbet på et prosjekt hvor målet var å studere skade på komponenter i superledende magneter
i Large Hadron Collider (LHC) grunnet protonstråling.
Arbeidet innebar å ta del i design av eksperimentelle oppsett,
simulering av vekselvirkninger mellom ladde partikler og faste stoffer og
gjennomføring av eksperimenter på kritiske magnetkomponenter i LHC.
}
\cventry{2015}{Sommerjobb i Silicon Labs}{}{}{}{
Implentering av bluetoothstøtte for Silicon Labs mikrokontrollere på
\Colorhref[linkColor]{https://www.mbed.com/en/}{Mbed} platformen.
I tillegg ble eksempelgalleriet for \emph{Happy Gecko Starter Kit} utvidet
ved å implementere det snake-lignende spillet
\Colorhref[linkColor]{https://developer.mbed.org/teams/SiliconLabs/code/Hungry_gecko/graph}{Hungry Gecko Game}
}
\cventry{2014}{Sommerjobb W{\"a}rtsil{\"a}}{}{}{}{Lage grafiske bilder for kontrollsystemapplikasjoner på lasteskip.}
\cventry{2014}{Sommerjobb i Polytec}{}{}{}{Forskningsassistent på et prosjekt
angående overflater med anti-is egenskaper.}
\cventry{2013}{Sommerjobb i Gassco}{}{}{}{Statistisk analyse av historiske produksjonsdata
og sammenligning mot eksisterende prognosemodeller.}
\cventry{2011 - 2013}{Studentassistent i ulike kurs ved NTNU}{}{}{}{
Veiledning av en gruppe på ca. 20 studenter under regneøvinger.
}

\section{Utdanning}
\cvitem{2016 - 2021}{Doktorgrad ved institutt for fysikk ved NTNU: \emph{Atomistic Modeling of Precipitate Formation in Aluminium Alloys}}
\cvitem{2011 - 2016}{Sivilingeniør i Fysikk og Matematikk ved NTNU (Norges teknisk-naturvitenskapelige universitet)}
\cvitem{2010 - 2011}{Førstegangstjeneste i HM Kongens Gardes Musikk og drill kompani.}
\cvitem{2007 - 2010}{Studiespesialisering ved Skeisvang vgs.}


\section{Tekniske ferdigheter}
\cvitem{Programmering}{C/C++, Fortran, MATLAB, Python, JavaScript, Go}
\cvitem{Skytjenester}{Azure, Google Cloud}
\cvitem{IT infrastruktur}{Kubernetes, Docker, Terraform, Kustomize}
\cvitem{Databaser}{SQL, SparQL, GraphDB}
\cvitem{Informasjon}{Resource Description Framework (RDF), Common Information Model (CIM)}
\cvitem{Web}{HTMX}
\cvitem{Tekstbehandling}{\LaTeX, MS Office}
\section{Språk}
\cvitem{Norsk}{Morsmål}
\cvitem{Engelsk}{Flytende}
\cvitem{Tysk}{Grunnleggende}
\cvitem{Fransk}{Grunnleggende (A1)}

%\section{Referanser}
%\cvitem{Veileder Ph.D}{Jaakko Akola \newline jaakko.akola@ntnu.no}

% Publication list
\nocite{*}
\bibliographystyle{unsrt}
\renewcommand{\refname}{Publikasjoner}
{\footnotesize \bibliography{conference.bib}}

\end{document}
