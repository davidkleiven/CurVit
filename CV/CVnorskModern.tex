\documentclass[verdana]{moderncv}

\usepackage[scale=0.8]{geometry}
\moderncvstyle{classic}
\moderncvcolor{blue}
\usepackage{filecontents}
\usepackage{fontspec}
\defaultfontfeatures{Ligatures=TeX}
\definecolor{linkColor}{rgb}{0.156 0.546 0.742}

\newcommand\Colorhref[3][cyan]{\href{#2}{\small\color{#1}#3}}
\makeatletter
\renewcommand*{\bibliographyitemlabel}{\@biblabel{\arabic{enumiv}}}
\makeatother

\begin{filecontents}{conference.bib}
@article{kleiven2020precipitate,
  title={Precipitate Formation in Aluminum Alloys: Multi-scale Modeling Approach},
  author={Kleiven, David and Akola, Jaakko},
  journal={Acta Materialia},
  year={2020},
  publisher={Elsevier}
  }

  @article{chang2019clease,
  title={CLEASE: a versatile and user-friendly implementation of cluster expansion method},
  author={Chang, Jin Hyun and Kleiven, David and Melander, Marko and Akola, Jaakko and Garcia-Lastra, Juan Maria and Vegge, Tejs},
  journal={Journal of Physics: Condensed Matter},
  volume={31},
  number={32},
  pages={325901},
  year={2019},
  publisher={IOP Publishing}
}

  @article{kleiven2019atomistic,
  title={Atomistic simulations of early stage clusters in AlMg alloys},
  author={Kleiven, David and {\O}deg{\aa}rd, Olve L and Laasonen, Kari and Akola, Jaakko},
  journal={Acta Materialia},
  volume={166},
  pages={484--492},
  year={2019},
  publisher={Elsevier}
}


  @inproceedings{kleiven2016experimental,
  title={Experimental Setup to Measure the Damage Limits of Superconducting Magnets due to Beam Impact at CERN's HiRadMat Facility},
  author={Kleiven, David and Auchmann, Bernhard and Raginel, Vivien and Schmidt, Ruediger and Verweij, Arjan and Wollmann, Daniel},
  booktitle={7th International Particle Accelerator Conference (IPAC'16), Busan, Korea, May 8-13, 2016},
  pages={4200--4202},
  year={2016},
  organization={JACOW, Geneva, Switzerland}
  }

  @inproceedings{raginel2016degradation,
    title={Degradation of the insulation of the LHC main dipole cable when exposed to high temperatures},
    author={Raginel, Vivien and Auchmann, Bernhard and Kleiven, David and Schmidt, Ruediger and Verweij, Arjan and Wollmann, Daniel},
    booktitle={7th International Particle Accelerator Conference (IPAC'16), Busan, Korea, May 8-13, 2016},
    pages={1186--1189},
    year={2016},
    organization={JACOW, Geneva, Switzerland}
  }

\end{filecontents}

\firstname{David}
\lastname{Kleiven}
\title{Curriculum Vitae}
\address{Edgar B. Schieldropsvei 100B}{7033 Trondheim}
\mobile{93060916}
\email{davidkleiven446@gmail.com}
\photo[64pt]{../imageCrop.jpg}

\begin{document}
\makecvtitle

\section{Utdanning}
\cvitem{2016 - }{Doktorgrad ved institutt for fysikk ved NTNU: \emph{Atomstisk simulering av presipitatsdannelse i aluminiumslegeringer}}
\cvitem{2011 - 2016}{Sivilingeniør i Fysikk og Matematikk ved NTNU (Norges teknisk-naturvitenskapelige universitet)}
\cvitem{2010 - 2011}{Førstegangstjeneste i HM Kongens Gardes Musikk og drill kompani.}
\cvitem{2007 - 2010}{Studiespesialisering ved Skeisvang vgs.}

\section{Arbeidserfaring}
\cventry{2015 - 2016}{Teknisk student ved CERN}{}{}{}{
Jobbet på et prosjekt hvor målet var å studere skade på komponenter i superledende magneter
i Large Hadron Collider (LHC) grunnet protonstråling.
Arbeidet innebar å ta del i design av eksperimentelle oppsett,
simulering av vekselvirkninger mellom ladde partikler og faste stoffer og
gjennomføring av eksperimenter på kritiske magnetkomponenter i LHC.
}
\cventry{2015}{Sommerjobb i Silicon Labs}{}{}{}{
Implentering av bluetoothstøtte for Silicon Labs mikrokontrollere på
\Colorhref[linkColor]{https://www.mbed.com/en/}{Mbed} platformen.
I tillegg ble eksempelgalleriet for \emph{Happy Gecko Starter Kit} utvidet
ved å implementere det snake-lignende spillet
\Colorhref[linkColor]{https://developer.mbed.org/teams/SiliconLabs/code/Hungry_gecko/graph}{Hungry Gecko Game}
}
\cventry{2014}{Sommerjobb W{\"a}rtsil{\"a}}{}{}{}{Lage grafiske bilder for kontrollsystemapplikasjoner på lasteskip.}
\cventry{2014}{Sommerjobb i Polytec}{}{}{}{Forskningsassistent på et prosjekt
angående overflater med anti-is egenskaper.}
\cventry{2013}{Sommerjobb i Gassco}{}{}{}{Statistisk analyse av historiske produksjonsdata
og sammenligning mot eksisterende prognosemodeller.}
\cventry{2011 - 2013}{Studentassistent i ulike kurs ved NTNU}{}{}{}{
Veiledning av en gruppe på ca. 20 studenter under regneøvinger.
}

\section{Tekniske ferdigheter}
\cvitem{Programmering}{C/C++, Fortran, MATLAB, Python, JavaScript, Go}
\cvitem{Tekstbehandling}{\LaTeX, MS Office}

\section{Språk}
\cvitem{Norsk}{Morsmål}
\cvitem{Engelsk}{Flytende}
\cvitem{Tysk}{Grunnleggende}
\cvitem{Fransk}{Grunnleggende}

\section{Referanser}
\cvitem{Veileder Ph.D}{Jaakko Akola \newline jaakko.akola@ntnu.no}

% Publication list
\nocite{*}
\bibliographystyle{unsrt}
\renewcommand{\refname}{Publikasjoner}
{\footnotesize \bibliography{conference.bib}}

\end{document}
